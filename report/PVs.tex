\documentclass[12pt,a4paper]{article}
%\usepackage[latin1]{inputenc}
\usepackage[francais]{babel}
\usepackage[T1]{fontenc}
\usepackage{amsmath}
\usepackage{amsfonts}
\usepackage{amssymb}
\usepackage{graphicx}
\usepackage[utf8x]{inputenc}
\usepackage[left=2cm,right=2cm,top=2cm,bottom=2cm]{geometry}
\usepackage{color}
\usepackage{url}
\title{Génie logiciel et gestion de projet \\ PV des réunions}
\author{Groupe 5}

\begin{document}
\maketitle

\newpage
\section{1ère réunion : 20/10/2014}
	
	\subsection{Présences}
		Tous les membres étaient présents à la réunion.	
	
	\subsection{Coordonnées des membres du groupe}

		\begin{tabular}{|lll|}
			\hline
			Nom & GSM & Mail \\
			\hline
			\hline
			Ségolène Rogge & 0475462893 & segolenerogge@gmail.com \\
			Martin Cornil & 0477962597 & macornil94@gmail.com \\
			Antoine Depuydt & 0473796629 & antoinedp@live.be \\
			Antoine Carpentier & 0494949396 & antoine.carpentier.info@gmail.com \\
			Daniele Bonatto & 0473863312 & itichi@gmail.com \\
			Franklin Laureles & 0474661229 & f.laureles@gmail.com \\
			\hline
		\end{tabular}

	\subsection{Points abordés}
		
		\subsubsection{Azendoo}
			Azendoo a été choisi pour être le moyen de communication principal entre les membres du groupes. Tous le monde y est maintenant inscrit, le travail peut commencer.
		\subsubsection{Git}
			Mail d'invitation reçue par tous les membres
		\subsubsection{Quels outils graphiques envisager ?}
			\begin{itemize}
				\item Swing
				\item Jambi
				\item OpenGL
				\item OpenCL
				\item Java3D
				\item Povray
			\end{itemize}
		\subsubsection{Rappels sur quelques principes de l'Extreme Programming}
			\begin{itemize}
				\item Pair programming
				\item Tests unitaires
				\item Communication entre membres
			\end{itemize}
		\subsubsection{Discussion sur les différentes histoires}
			\paragraph{Ordre d'importance}
				1 - 2 - 4 - 5 - 3 - 6
			\paragraph{Exigences pour l'histoire 1 \\}
				De la plus importante à la moins importante, on a :
				\begin{itemize}
					\item Fenêtre GUI
					\item Configuration de la maison
					\item Affichage 2D/3D
					\item Déplacement dans un projet
					\item Gestionnaire de projet
					\item Reconfiguration d'un projet
					\item Sauvegarde automatique
				\end{itemize}
	\subsection{Prochaine réunion}
		\subsubsection{Date}
			Mercredi 29/10 à 10h.
		\subsubsection{À faire}
			\begin{itemize}
				\item Installer eclipse avec tous les plugins nécessaires (voir pdf)
				\item Configurer GIT
				\item Faire une première ébauche de diagramme UML
				\item Relire ce qu'est l'Extreme Programming
			\end{itemize}

\newpage
\section{2ème réunion : 29/10/2014}
	\subsection{Présences}
		Tous les membres étaient présents à la réunion.
	\subsection{Points abordés}
		\subsubsection{Outils choisis pour la réalisation du projet}
			\begin{itemize}
				\item jMonkey pour la 3D
				\item Swing pour l'interface graphique
			\end{itemize}
			Remarque : petit problèmes de threads avec Swing, à régler.
		\subsubsection{UML}
			Mise en commun des différents graphes UML réalisés par le groupe. Choix final : reprendre l'UML de Martin en y ajoutant des détails de celui de Daniele. Martin le pushera dès qu'il sera fini.
		\subsubsection{Répartition des tâches}
			\begin{itemize}
				\item Martin : création de la fenêtre graphique
				\item Franklin : se renseigner sur la serialisation pour les sauvegardes (automatiques) $\rightarrow$ plus efficace que .obj ? Sauvegardes en cascade ?
				\item Daniele : se renseigner sur les threads et swing
				\item Antoine D. : affichage 2D/3D
				\item Antoine C. : ouvrir/modifier les anciens fichiers ou en créer un nouveau
				\item Ségolène : SQL + se renseigner sur les tests unitaire pour l'interface graphique
			\end{itemize}
		\subsubsection{Remarque}
			\begin{itemize}
				\item Example pour la serialisation : \url{www.tutorialspoint.com/java/java_serialization.htm}
			\end{itemize}
			
		\subsection{Prochaine réunion}
			\subsubsection{Date}
				Lundi 3/11 à 15h.
			\subsubsection{À faire}
				\begin{itemize}
					\item Priorité sur les parties de Martin et Antoine D.
					\item Perfectionner ses connaissances sur Swing, jMonkey et l'UML (pour ceux qui ne le connaissent pas bien)
					\item S'informer sur MVC
				\end{itemize}
			
\newpage
\section{3ème réunion}
	\subsection{Présences}
		Tous les membres étaient présents à la réunion.
	\subsection{Points abordés}
		\subsubsection{Eclipse et jMonkey}
			Installation sur tous les ordinateurs des membres.
		\subsubsection{Fenêtre graphique}
			Problème d'affichage : les parties 2D et 3D sont "écrasées" si on les affiche sur un seul écran $\rightarrow$ choix : les afficher sur des écrans différents (appuyer sur un bouton pour passer de l'un à l'autre).
		\subsubsection{Sauvegarde des objets}
			Préférence pour les bases de données (en ORM) plutôt que pour la serialization, qui semble plus complexe.			
			
			Pour enregister les objets (les murs, etc...), on pourrait utiliser hibernate ou ormlite, si les assistants sont d'accord (à vérifier).
			%(table projet, table scène -> liste de référence vers des objets qui ont des coordonnées vers leurs parents.)
			
			Utiliser le type blob dans la base de données pour stocker les objets ou les stocker sous forme de fichier dans un ordinateur ?
		\subsubsection{Création de la maison}
			Il faut tout d'abord créer le terrain, puis les murs afin de délimiter la maison. Un système de node (inclu dans jMonkey) pourrait être utilisé pour attacher chaque objet à son étage ou aux objets sur lequel il est posé/lié. Pour le modèle, regarder du coté des design patterns.
			
	\subsection{Prochaine réunion}
		\subsubsection{Date}
			Lundi 10/11 à 15h.
			
		\subsubsection{À faire}
			\begin{itemize}
				\item Antoine D. : faire le terrain
				\item Daniele et Ségolène : créer les classes principales du modèle + créer/modifier les murs
				\item Franklin et Antoine C.: sauvegardes/investiguer hibernate et ormline et les formats des objets
				\item Martin : affichage 2D/3D 
			\end{itemize}

\newpage
\section{4ème réunion}
	\subsection{Présences}
		Tous les membres étaient présents à la réunion.
	\subsection{Points abordés}
	
		\subsubsection{Changement outils conversation/gestion de projet}
			Passage de Azendoo vers trello.com.
		\subsubsection{Histoire pour l'itération 2}
			Histoire 5 : il faudra convertir les objets importés pour qu'ils aient le même format que les objets que nous créerons. (JME gère déjà les formats OBJ, DAE, 3DS et KMZ)
			
		\subsubsection{ORM}
			Choix : ORMlite parce que hibernate utilise des xml $\rightarrow$ hyper chiant :P
			
		\subsubsection{Queue dans la flycam}
			Quand on va créer un mur, on enregistre les points jusqu'à ce que l'utilisateur dise "crée le mur" et on réccupère les points qu'il faut dans la queue. Pour le différencier d'un drag and drop, on va vérifier que les coordonnées de l'event cliquer et celui de l'event relacher soient les mêmes.
			
		\subsubsection{Épaisseur des murs}
			Détection si mur externe -> pour pas dépasser de la dalle.
			
			Si mur interne, on dessine le mur de manière à ce que les coordonnées données soient au milieu de l'épaisseur.
			
		\subsubsection{Database}
			Créer une relation many-to-many pour gérer les bases de données $\rightarrow$ encapsulation

	\subsection{Prochaine réunion}
		\subsubsection{Date}
			Lundi 17/11 à 15h. 
			
		\subsubsection{À faire}
			\begin{itemize}
				\item Martin, Daniele et Ségolène : Écrire le contrôleur
				\item Antoine C. et Franklin : Enregistrer les objets ($+$ sauvegardes automatiques)
				\item Antoine C. : Diagramme UML de la database (ce qu'on va sauver, quelles classes de travail, etc)
				\item Antoine D. et Martin : Gestion des collisions (possibilité de cliquer sur un objet) 
				\item Déplacer les murs (?) \\
			\end{itemize}
			
			 !! Ne pas oublier les tests !!

\newpage
\section{5ème réunion}
	\subsection{Présences}
		Tous les membres étaient présents à la réunion.
	\subsection{Points abordés}
		\subsubsection{Controleur}
			Explication sur le fonctionnement du controleur, et l'interaction avec la vue et le modèle.
		\subsubsection{Database}
			Résumé sur le fonctionnement de la database.
		\subsubsection{Système de sauvegarde automatique}
			Mettre un timer dans le thread de l'affichage ou faire un nouveau thread ? Dans le thread de l'affichage, pourrait faire laguer $\rightarrow$ nouveau thread.
		\subsubsection{Histoire 1}
			Ce qu'il reste à faire pour compléter la première itération :
			\begin{itemize}
				\item Merge du controleur et de la database
				\item Sauvegardes automatiques
				\item Ouverture/modification d'un ancien projet
				\item Configurer la maison : permettre à l'utilisateur de choisir la taille de son terrain à la création d'un projet
				\item Outils draw et pull-up
				\item Faire la javadoc
			\end{itemize}
			
	\subsection{Prochaine réunion}
		\subsubsection{Date}
			Mardi 18/11 12h pour terminer l'itération 1 tous ensemble.
			
		\subsubsection{À faire}
			Commencer à coder ce qu'il reste.

\end{document}

